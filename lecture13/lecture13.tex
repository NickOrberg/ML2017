\documentclass[10pt]{beamer}

\usepackage[T2A]{fontenc}
\usepackage[utf8]{inputenc}
\usepackage[russian,english]{babel}

\usefonttheme[onlymath]{serif}

\usetheme[progressbar=frametitle]{metropolis}
\usepackage{appendixnumberbeamer}

\usepackage{booktabs}
\usepackage[scale=2]{ccicons}

\usepackage{pgfplots}
\usepgfplotslibrary{dateplot}

\usepackage{xspace}
\newcommand{\themename}{\textbf{\textsc{metropolis}}\xspace}
\newcommand{\TODO}[1]{\textbf{\textcolor{red}{TODO: #1}}}

\date{}
\author{Екатерина Тузова}


\title{Лекция 13}
\subtitle{Композиции алгоритмов}

\begin{document}

\maketitle

\section{Разбор летучки}

\section{Мотивирующий пример}

\begin{frame}{Пример}
  \TODO{пример}  
\end{frame}


\begin{frame}{Определение композиции}
  ${X^l = (x_i, y_i)_{i = 1}^l}$ -- обучающая выборка\\
  $b:X \rightarrow R$ --- базовый алгоритм\\
  $C:R \rightarrow Y$ --- решающее правило\\
  $R$ --- пространство оценок.
  \bigbreak
  $a(x) = C(b(x))$\\
\end{frame}

\begin{frame}{Определение композиции}
  Композиция базовых алгоритмов $b_1, \dots, b_t$\\
  \bigbreak
  $a(x) = C(F(b_1(x), \dots, b_t(x)))$\\
  $F: R^T \rightarrow R$ -- корректирующая операция
\end{frame}

\begin{frame}{Примеры}
  \begin{enumerate}
    \item Простое голосование\\
      $$F(b_1(x), \dots, b_t(x)) = \frac{1}{T} \sum\limits_{t=1}^{T} b_t(x)$$
    \item Взвешенное голосование\\
      $$F(b_1(x), \dots, b_t(x)) = \sum\limits_{t=1}^{T} \alpha_t b_t(x)$$
    \item Смесь алгоритмов\\
      $$F(b_1(x), \dots, b_t(x)) = \sum\limits_{t=1}^{T} g_t(x) b_t(x)$$      
  \end{enumerate}
\end{frame}

{\foot{Boosting}
\begin{frame}{Бустинг}
  $Y = \{\pm 1\}$, $b_t: X\rightarrow \{-1, 0, +1\}$, $C(b) = \sign(b)$\\
  $b_t(x) = 0$ -- отказ от классификации\\
  \bigbreak
  \pause
  $a(x) = \sign(\sum\limits_{t=1}^{T} \alpha_t b_t(x))$\\
  \bigbreak
  Функционал качества композиции:\\
  $Q_T = \sum\limits_{i=1}^l [y_i \sum\limits_{t=1}^{T} \alpha_t b_t(x) < 0 ]$
\end{frame}
}

\begin{frame}[standout]
  Вопросы?
\end{frame}

\appendix

\begin{frame}\frametitle{На следующей лекции}
	\begin{itemize}
    	\item[--] Функционалы качества
    	\item[--] Неравенство Хефдинга
    	\item[--] Близость гипотез
    	\item[--] Неравенство Вапника-Червоненкиса
    	\item[--] Генерация модельных данных    	    	
	\end{itemize}
\end{frame}
\end{document}

\end{document}